% Options for packages loaded elsewhere
\PassOptionsToPackage{unicode}{hyperref}
\PassOptionsToPackage{hyphens}{url}
\PassOptionsToPackage{dvipsnames,svgnames,x11names}{xcolor}
%
\documentclass[
  authoryear,
  preprint,
  3p]{elsarticle}

\usepackage{amsmath,amssymb}
\usepackage{iftex}
\ifPDFTeX
  \usepackage[T1]{fontenc}
  \usepackage[utf8]{inputenc}
  \usepackage{textcomp} % provide euro and other symbols
\else % if luatex or xetex
  \usepackage{unicode-math}
  \defaultfontfeatures{Scale=MatchLowercase}
  \defaultfontfeatures[\rmfamily]{Ligatures=TeX,Scale=1}
\fi
\usepackage{lmodern}
\ifPDFTeX\else  
    % xetex/luatex font selection
\fi
% Use upquote if available, for straight quotes in verbatim environments
\IfFileExists{upquote.sty}{\usepackage{upquote}}{}
\IfFileExists{microtype.sty}{% use microtype if available
  \usepackage[]{microtype}
  \UseMicrotypeSet[protrusion]{basicmath} % disable protrusion for tt fonts
}{}
\makeatletter
\@ifundefined{KOMAClassName}{% if non-KOMA class
  \IfFileExists{parskip.sty}{%
    \usepackage{parskip}
  }{% else
    \setlength{\parindent}{0pt}
    \setlength{\parskip}{6pt plus 2pt minus 1pt}}
}{% if KOMA class
  \KOMAoptions{parskip=half}}
\makeatother
\usepackage{xcolor}
\setlength{\emergencystretch}{3em} % prevent overfull lines
\setcounter{secnumdepth}{5}
% Make \paragraph and \subparagraph free-standing
\ifx\paragraph\undefined\else
  \let\oldparagraph\paragraph
  \renewcommand{\paragraph}[1]{\oldparagraph{#1}\mbox{}}
\fi
\ifx\subparagraph\undefined\else
  \let\oldsubparagraph\subparagraph
  \renewcommand{\subparagraph}[1]{\oldsubparagraph{#1}\mbox{}}
\fi


\providecommand{\tightlist}{%
  \setlength{\itemsep}{0pt}\setlength{\parskip}{0pt}}\usepackage{longtable,booktabs,array}
\usepackage{calc} % for calculating minipage widths
% Correct order of tables after \paragraph or \subparagraph
\usepackage{etoolbox}
\makeatletter
\patchcmd\longtable{\par}{\if@noskipsec\mbox{}\fi\par}{}{}
\makeatother
% Allow footnotes in longtable head/foot
\IfFileExists{footnotehyper.sty}{\usepackage{footnotehyper}}{\usepackage{footnote}}
\makesavenoteenv{longtable}
\usepackage{graphicx}
\makeatletter
\def\maxwidth{\ifdim\Gin@nat@width>\linewidth\linewidth\else\Gin@nat@width\fi}
\def\maxheight{\ifdim\Gin@nat@height>\textheight\textheight\else\Gin@nat@height\fi}
\makeatother
% Scale images if necessary, so that they will not overflow the page
% margins by default, and it is still possible to overwrite the defaults
% using explicit options in \includegraphics[width, height, ...]{}
\setkeys{Gin}{width=\maxwidth,height=\maxheight,keepaspectratio}
% Set default figure placement to htbp
\makeatletter
\def\fps@figure{htbp}
\makeatother

\makeatletter
\@ifpackageloaded{caption}{}{\usepackage{caption}}
\AtBeginDocument{%
\ifdefined\contentsname
  \renewcommand*\contentsname{Table of contents}
\else
  \newcommand\contentsname{Table of contents}
\fi
\ifdefined\listfigurename
  \renewcommand*\listfigurename{List of Figures}
\else
  \newcommand\listfigurename{List of Figures}
\fi
\ifdefined\listtablename
  \renewcommand*\listtablename{List of Tables}
\else
  \newcommand\listtablename{List of Tables}
\fi
\ifdefined\figurename
  \renewcommand*\figurename{Figure}
\else
  \newcommand\figurename{Figure}
\fi
\ifdefined\tablename
  \renewcommand*\tablename{Table}
\else
  \newcommand\tablename{Table}
\fi
}
\@ifpackageloaded{float}{}{\usepackage{float}}
\floatstyle{ruled}
\@ifundefined{c@chapter}{\newfloat{codelisting}{h}{lop}}{\newfloat{codelisting}{h}{lop}[chapter]}
\floatname{codelisting}{Listing}
\newcommand*\listoflistings{\listof{codelisting}{List of Listings}}
\makeatother
\makeatletter
\makeatother
\makeatletter
\@ifpackageloaded{caption}{}{\usepackage{caption}}
\@ifpackageloaded{subcaption}{}{\usepackage{subcaption}}
\makeatother
\journal{Journal of the Acoustical Society of America}
\ifLuaTeX
  \usepackage{selnolig}  % disable illegal ligatures
\fi
\usepackage[]{natbib}
\bibliographystyle{elsarticle-harv}
\usepackage{bookmark}

\IfFileExists{xurl.sty}{\usepackage{xurl}}{} % add URL line breaks if available
\urlstyle{same} % disable monospaced font for URLs
\hypersetup{
  pdftitle={SPI - Defining bespoke and archetypal context-dependent Soundscape Perception Indices},
  pdfauthor={Andrew Mitchell; Francesco Aletta; Tin Oberman; Jian Kang},
  pdfkeywords={keyword1, keyword2},
  colorlinks=true,
  linkcolor={blue},
  filecolor={Maroon},
  citecolor={Blue},
  urlcolor={Blue},
  pdfcreator={LaTeX via pandoc}}

\setlength{\parindent}{6pt}
\begin{document}

\begin{frontmatter}
\title{SPI - Defining bespoke and archetypal context-dependent
Soundscape Perception Indices}
\author[1]{Andrew Mitchell%
\corref{cor1}%
}
 \ead{andrew.mitchell.18@ucl.ac.uk} 
\author[1]{Francesco Aletta%
%
}
 \ead{f.aletta@ucl.ac.uk} 
\author[1]{Tin Oberman%
%
}
 \ead{t.oberman@ucl.ac.uk} 
\author[]{Jian Kang%
%
}
 \ead{j.kang@ucl.ac.uk} 

\affiliation[1]{organization={University College London, Institute for
Environmental Design \& Engineering},,postcodesep={}}

\cortext[cor1]{Corresponding author}




        
\begin{abstract}
The soundscape approach provides a basis for considering the holistic
perception of sound environments, in context. While steady advancements
have been made in methods for assessment and analysis, a gap exists for
comparing soundscapes and quantifying improvements in the
multi-dimensional perception of a soundscape. To this end, there is a
need for the creation of single value indices to compare soundscape
quality which incorporate context, aural diversity, and specific design
goals for a given application. Just as a variety of decibel-based
indices have been developed for various purposes (e.g.~LAeq, LCeq, L90,
Lden, etc.), the soundscape approach requires the ability to create
novel indices for different uses, but which share a common language and
understanding. We therefore propose a unified framework for creating
both bespoke and standardised single index measures of soundscape
perception based on the soundscape circumplex model, allowing for new
metrics to be defined in the future. The implementation of this
framework is demonstrated through the creation of a public spaced
typology-based index using data collected under the SSID Protocol, which
was designed specifically for the purpose of defining soundscape
indices. Indices developed under this framework can enable a broader and
more efficient application of the soundscape approach.
\end{abstract}





\begin{keyword}
    keyword1 \sep 
    keyword2
\end{keyword}
\end{frontmatter}
    
\section{Introduction}\label{introduction}

The EU Green Paper on Future Noise Policy indicates that 80 million EU
citizens are suffering from unacceptable environmental noise levels ,
according to the WHO recommendation \citep{Berglund1999Guidelines} and
the social cost of transport noise is 0.2-2\% of total GDP. The
publication of the EU Directive Relating to the Assessment and
Management of Environmental Noise (END)
\citep{EuropeanUnion2002Directive} in 2002 has led to major actions
across Europe, with reducing noise levels as the focus, for which
billions of Euros are being spent. However, it is widely recognised that
only reducing sound level is not always feasible or cost-effective, and
more importantly, with only \textasciitilde30\% of environmental noise
annoyance depending on facets of parameters such as acoustic energy
\citep{Guski1997Psychological}, sound level reduction will not
necessarily lead to improved quality of life.

Soundscape creation, separate from noise control engineering, is about
the relationships between human physiology, perception, the sound
environment, and its social/cultural context \citep{Kang2006Urban}.
Soundscape research represents a paradigm shift in that it combines
physical, social, and psychological approaches and considers
environmental sounds as a `resource' rather than `waste'
\citep{Kang2016Soundscape} relating to perceptual constructs rather than
just physical phenomena. However, the current research is still at the
stage of describing and identifying the problems and tends to be
fragmented and focussed on only special cases e.g.~subjective
evaluations of soundscapes for residential areas
\citep{SchulteFortkamp2013Introduction}. In the movement from noise
control to soundscape creation \citep{Aletta2015Soundscape}, a vital
step is the standardisation of methods to assess soundscape quality.

The Decibel (dB) is the earliest and most commonly used scientific index
measuring sound level. To represent the overall level of sound with a
single value on one scale, as the Decibel index does, is often
desirable. For this purpose, a number of different values representing
sounds at various frequencies must be combined. Several frequency
weighting networks have been developed since the 1930s, considering
typical human responses to sound based on equal-loudness-level contours
\citep{viii} and, among them, the A-weighting network, with resultant
decibel values called dBA, has been commonly used in almost all the
national/international regulations \citep{ix}. However, there have been
numerous criticisms on its effectiveness \citep{x} as the correlations
between dBA and perceived sound quality (e.g.~noise annoyance) are often
low \citep{xi}.

Another set of indices is psychoacoustic magnitudes, including loudness,
fluctuation strength or roughness, sharpness, and pitch strength,
development with sound quality studies of industrial products since the
1980's \citep{xii}. These emerged when it was conceived that acoustic
emissions had further characteristics than just level \citep{ciii}. But
while psychoacoustic magnitudes have been proved to be successful for
the assessment of product sound quality \citep{xiv}, in the field of
environmental acoustics, their applicability has been limited
\citep{xv}, since a significant feature of environmental acoustics is
that there are multiple/dynamic sound sources.

Attendant with the transition from a noise reduction to soundscape
paradigm is an urgent need for developing appropriate indices for
soundscape, rather than continuously using dBA \citep{xvi}.

\subsection{The need for Soundscape
Indices}\label{the-need-for-soundscape-indices}

Soundscape studies strive to understand the perception of a sound
environment, in context, including acoustic, (non-acoustic)
environmental, contextual, and personal factors. These factors combine
together to form a person's soundscape perception in complex interacting
ways \citep{Berglund2006Soundscape}. Humans and soundscapes have a
dynamic bidirectional relationship - while humans and their behaviour
directly influence their soundscape, humans and their behaviour are in
turn influenced by their soundscape
\citep{Erfanian2019Psychophysiological}.

When applied to urban sound and specifically to noise pollution, the
soundscape approach introduces three key considerations beyond
traditional noise control methods:

\begin{enumerate}
\def\labelenumi{\arabic{enumi}.}
\tightlist
\item
  considering all aspects of the environment which may influence
  perception, not just the sound level and spectral content;
\item
  an increased and integrated consideration of the varying impacts which
  different sound sources and sonic characteristics have on perception;
  and
\item
  a consideration of both the positive and negative dimensions of
  soundscape perception.
\end{enumerate}

This approach can enable better outcomes by identifying positive
soundscapes (in line with the END's mandate to `preserve environmental
noise quality where it is good' \citep{EuropeanUnion2002Directive}),
better identify specific sources of noise which impact soundscape
quality and pinpoint the characteristics which may need to be decreased,
and illuminate alternative methods which could be introduced to improve
a soundscape where a reduction of noise is impractical
\citep{Fiebig2018Does, Kang2018Impact}. These can all lead to more
opportunities to truly improve a space by identifying the causes of
positive soundscapes, while also potentially decreasing the costs of
noise mitigation by offering more targeted techniques and alternative
approaches.

The traditional focus on noise levels alone fails to capture the
complexity of soundscape perception, which encompasses a multitude of
factors beyond mere sound pressure levels. Factors such as the presence
of natural or human-made sounds, their temporal patterns, and the
overall contextual meaning ascribed to these sounds all contribute to
the holistic perception of a soundscape. Consequently, there is a
pressing need for the development of robust indices that can encapsulate
this multi-dimensional nature of soundscape perception, enabling
comparative evaluations and informing targeted interventions to enhance
the overall quality of acoustic environments
\citep{Chen2024Interventions}.

Across both the visual and the auditory domain, research has suggested
that a disconnect exists between the physical metrics used to describe
urban environments and how they are perceived
\citep{Kruize2019Exploring, Yang2005Acoustic}. In addition, this
disconnect can be extended further into how these environments influence
the health and well-being of their users. To gain a better understanding
of these spaces and their immpacts on people who work and live in
cities, we must create assessment methods and metrics which go beyond
merely characterising the physical environment and instead translate
through the user's perception \citep{Mitchell2022Predictive}.

\subsection{Note on Terminology}\label{note-on-terminology}

Before delving into the core discussion, it is crucial to establish a
clear understanding of the terminology employed in the realm of
soundscape evaluation.

The soundscape community is undergoing a period of increased
methodological standardization in order to better coordinate and
communicate the findings of the field. This process has resulted in many
operational tools designed to assess and understand how sound
environments are perceived and apply this to shape modern noise control
engineering approaches. Important topics which have been identified
throughout this process are soundscape `descriptors', `indicators', and
`indices'. \citet{Aletta2016Soundscape} defined soundscape descriptors
as `measures of how people perceive the acoustic environment' and
soundscape indicators as `measures used to predict the value of a
soundscape descriptor'. Soundscape indices can then be defined as
`single value scales derived from either descriptors or indicators that
allow for comparison across soundscapes' \citep{Kang2019Towards}.

Soundscape indicators refer to measurable aspects or attributes of a
soundscape, such as loudness, tonal characteristics, or spectral
content, which can be quantified through objective measurements or
signal processing techniques. In contrast, soundscape descriptors are
qualitative representations of the perceived characteristics of a
soundscape, often derived from listener evaluations, subjective
assessments, or semantic differential scales \citep{ISO12913Part2}.

Indices, the primary focus of this article, are single numerical values
that combine multiple indicators or descriptors to provide a
comprehensive representation of the overall soundscape perception. These
indices serve as powerful tools for quantifying and comparing
soundscapes, enabling decision-makers and stakeholders to assess the
impact of interventions, monitor changes over time, and prioritize areas
for improvement.

\citep{Grinfeder2022What}

\subsection{Existing `Soundscape
Indices'}\label{existing-soundscape-indices}

While the field of soundscape research has witnessed substantial
progress, the development of standardized indices for evaluating and
comparing soundscapes across diverse contexts has been relatively
limited. Existing indices can be broadly seen as arising from two
domains: soundscape ecology and soundscape perception. It is worth
reviewing these indices to highlight how the framework proposed here is
fundamentally different in both concept and aim.

\subsubsection{Soundscape Ecology}\label{soundscape-ecology}

Within the realm of soundscape ecology, indices such as the Acoustic
Diversity Index (ADI) and Frequency-dependenty Acoustic Diversity Index
(FADI) \citep{Xu2023frequency} have been developed to quantify the
diversity and complexity of acoustic signals within a given soundscape.
These indices are particularly useful in ecological studies, providing
insights into the richness and diversity of biophonic (natural) and
anthrophonic (human-made) sound sources.

\textbf{\emph{Add additional information on ADI, FADI, NDSI, etc.}}

However, while these indices contribute valuable insights into the
ecological aspects of soundscapes, they do not directly address the
perceptual dimensions that are central to the soundscape approach
\citep{SchulteFortkamp2023Soundscapes}. The multi-dimensional nature of
soundscape perception, encompassing factors such as pleasantness,
eventfulness, and familiarity, necessitates a more comprehensive and
context-sensitive approach.

\subsubsection{Soundscape Perception}\label{soundscape-perception}

In the domain of soundscape perception, the Green Soundscape Index (GSI)
\citep{Kogan2018Green} has emerged as a notable attempt to quantify the
perceived quality of soundscapes, particularly in urban environments.
This index incorporates factors such as the presence and levels of
natural sounds, human-made sounds, and their respective contributions to
the overall soundscape perception.

The GSI is the ratio of the perceived extent of natural sounds (PNS) to
the perceived extent of traffic noise (PTN):

\[
GSI = \frac{<PNS>}{<PTN>}
\]

The GSI is noted to range between 1/5 and 5, with several ranges of
values given which correspond to general categories of the perceived
dominance of traffic noise.

While GSI represents a commendable effort to bridge the gap between
objective measurements and subjective perceptions, it remains limited in
its ability to capture the full complexity of soundscape perception
across diverse contexts. The intricate interplay between various sound
sources, their temporal patterns, and the specific context in which they
are experienced necessitates a more flexible and adaptable approach to
index development.

\subsection{Motivations \& Goals}\label{motivations-goals}

The primary motivation behind the development of the Soundscape
Perception Indices (SPIs) framework stems from the need to address the
existing gap in quantifying and comparing soundscape quality across
diverse contexts and applications. By creating a unified framework for
defining these indices, the aim is to facilitate a broader and more
efficient application of the soundscape approach in various domains,
such as urban planning, environmental management, acoustic design, and
policy development.

The overarching aim of this framework is to empower stakeholders,
decision-makers, and researchers with the ability to create tailored
indices that align with their specific objectives and design goals,
while simultaneously enabling cross-comparisons and benchmarking against
empirically-defined soundscape archetypes. This dual approach not only
acknowledges the context-dependent nature of soundscape perception but
also fosters a common language and understanding, facilitating knowledge
sharing and collaborative efforts within the field.

\emph{Ranking} - The ability to rank soundscapes based on their quality
is a key goal of the SPI framework. This ranking can be used to compare
soundscapes across different contexts, identify areas for improvement,
and prioritize interventions accordingly.

\emph{Standardisation} - The SPI framework aims to provide a
standardized approach for defining and calculating soundscape indices,
ensuring consistency and comparability across different applications and
domains. This standardization enables the development of best practices
and facilitates knowledge exchange within the field.

\section{Theoretical Background (of quantitative soundscape perception
measurements)}\label{theoretical-background-of-quantitative-soundscape-perception-measurements}

\subsection{Soundscape Circumplex \&
Projection}\label{soundscape-circumplex-projection}

SPI is grounded in the soundscape circumplex model
\citep{Axelsson2010principal, Axelsson2012Swedish}, a robust theoretical
foundation for understanding and representing the multi-dimensional
nature of soundscape perception. The reason for grounding the SPI into
the soundscape circumplex is that we have observed this model (and its
corresponding PAQs) to become the most prevalent one in soundscape
literature \citep{Aletta2023Adoption}. For the sake of supporting
standardization, we feel that we need the SPI to align to this model.

Method A is built on a series of descriptors referred to as the
Perceived Affective Quality (PAQ), proposed by
\citep{Axelsson2010principal}. These PAQs are based on the
pleasantness-activity paradigm present in research on emotions and
environmental psychology, in particular Russell's circumplex model of
affect \citep{Russell1980circumplex}. As summarised by Axelsson:
``Russell's model identifies two dimensions related to the perceived
pleasantness of environments and how activating or arousing the
environment is.''

To move the 8-item PAQ responses into the 2-dimensional circumplex
space, we use the projection method first presented in ISO 12913-3:2018.
This projection method and its associated formulae were recently updated
further in \citet{Aletta2024} to include a correction for the language
in which the survey was conducted. The formulae are as follows:

\[
% \begin{align*}
P_{ISO} = \frac{1}{\lambda_{pl}} \sum_{i=1}^{8} \cos \theta_i \cdot \sigma_i \\
E_{ISO} = \frac{1}{\lambda_{pl}} \sum_{i=1}^{8} \sin \theta_i \cdot \sigma_i 
% \end{align*}
\]

where \$\text{PAQ}\_i\$ is the response to the (i)th item of the PAQ.
The resulting (x) and (y) values are then used to calculate the polar
angle (\theta) and the radial distance (r) as follows:

\textbf{\emph{Add formulae for \(\theta\) and r}}

Using the angles derived in \citet{Aletta2024}, the following table is
used to convert the angles into the ISO 12913-3:2018 circumplex space:

By projecting specific soundscape perception responses into this
circumplex, it becomes possible to quantify their perceptual
characteristics.

\begin{verbatim}
Renaming PAQ columns.
Checking PAQ data quality.
Identified 109 samples to remove.
[6, 9, 13, 30, 32, 46, 190, 213, 229, 244, 296, 412, 413, 428, 464, 485, 655, 734, 739, 762, 766, 780, 1067, 1274, 1290, 1316, 1320, 1338, 1346, 1347, 1397, 1425, 1431, 1446, 1447, 1470, 1485, 1491, 1504, 1505, 1510, 1512, 1517, 1522, 1523, 1527, 1599, 1698, 1734, 1817, 1911, 1948, 2069, 2107, 2109, 2111, 2150, 2199, 2277, 2293, 2384, 2386, 2490, 2523, 2584, 2592, 2695, 2762, 2767, 2783, 2789, 2825, 2826, 2832, 2840, 2856, 2859, 2879, 2883, 2889, 2910, 2932, 2956, 2969, 3031, 3058, 3077, 3124, 3149, 3163, 3185, 3202, 3210, 3211, 3212, 3213, 3214, 3215, 3216, 3272, 3302, 3365, 3414, 3491, 3502, 3510, 3517, 3533, 3583]
\end{verbatim}

\begin{longtable}[]{@{}llllllllll@{}}
\toprule\noalign{}
& count & ISOPleasant & ISOEventful & pleasant & eventful & vibrant &
chaotic & monotonous & calm \\
\midrule\noalign{}
\endhead
\bottomrule\noalign{}
\endlastfoot
CarloV & 116 & 0.575 & 0.067 & 0.957 & 0.517 & 0.474 & 0.043 & 0.000 &
0.483 \\
SanMarco & 95 & 0.284 & 0.450 & 0.811 & 0.958 & 0.768 & 0.189 & 0.000 &
0.042 \\
PlazaBibRambla & 18 & 0.492 & 0.016 & 0.944 & 0.611 & 0.556 & 0.056 &
0.000 & 0.389 \\
CamdenTown & 105 & -0.022 & 0.408 & 0.467 & 0.914 & 0.410 & 0.505 &
0.029 & 0.057 \\
EustonTap & 96 & -0.118 & 0.237 & 0.312 & 0.771 & 0.240 & 0.531 & 0.156
& 0.073 \\
Noorderplantsoen & 97 & 0.559 & 0.467 & 0.969 & 0.979 & 0.948 & 0.031 &
0.000 & 0.021 \\
MarchmontGarden & 104 & 0.397 & 0.069 & 0.769 & 0.587 & 0.452 & 0.135 &
0.096 & 0.317 \\
MonumentoGaribaldi & 32 & 0.561 & 0.109 & 1.000 & 0.625 & 0.625 & 0.000
& 0.000 & 0.375 \\
TateModern & 152 & 0.467 & 0.312 & 0.928 & 0.862 & 0.789 & 0.072 & 0.000
& 0.138 \\
PancrasLock & 93 & 0.361 & 0.177 & 0.796 & 0.731 & 0.548 & 0.183 & 0.022
& 0.247 \\
TorringtonSq & 113 & 0.179 & 0.273 & 0.681 & 0.796 & 0.540 & 0.257 &
0.062 & 0.142 \\
RegentsParkFields & 107 & 0.709 & 0.043 & 1.000 & 0.570 & 0.570 & 0.000
& 0.000 & 0.430 \\
RegentsParkJapan & 89 & 0.783 & 0.137 & 0.989 & 0.719 & 0.708 & 0.011 &
0.000 & 0.281 \\
RussellSq & 145 & 0.585 & 0.169 & 0.952 & 0.703 & 0.662 & 0.041 & 0.007
& 0.290 \\
StPaulsCross & 66 & 0.454 & 0.242 & 0.909 & 0.773 & 0.712 & 0.061 &
0.030 & 0.197 \\
StPaulsRow & 72 & 0.332 & 0.200 & 0.833 & 0.750 & 0.625 & 0.125 & 0.042
& 0.208 \\
CampoPrincipe & 110 & 0.523 & -0.046 & 0.945 & 0.473 & 0.427 & 0.045 &
0.009 & 0.518 \\
MiradorSanNicolas & 28 & 0.387 & 0.146 & 0.964 & 0.679 & 0.643 & 0.036 &
0.000 & 0.321 \\
\end{longtable}

\subsubsection{Circumplex
Distribution}\label{sec-circumplex-distribution}

The circumplex is defined by two axes: \(P_{ISO}\) and \(E_{ISO}\),
which are limited to the range \([-1, +1]\). Typically, data in the
soundscape circumplex is treated as a combination of two independent
normal distributions, one for each axis. In some applications, this
approach is sufficient for capturing the distribution of soundscape
perception, however the method for calculating the SPI requires a more
precise approach. The independent normal distribution approach relies on
three key assumptions:

\begin{enumerate}
\def\labelenumi{\arabic{enumi}.}
\tightlist
\item
  The two axes are normally distributed.
\item
  The two axes are independent of each other.
\item
  The two axes are symmetrically distributed.
\end{enumerate}

While the first assumption is generally valid, the second and third
assumptions are not always met in practice. In particular, the
distribution of soundscape perception responses in the circumplex is
often characterised by a high degree of skewness, which can lead to
inaccuracies in the calculation of the SPI. Soundscape circumplex
distributions are most appropriately described as a bivariate
skew-normal distribution \citet{Azzalini2005} which accurately reflects
the relationship between the two dimensions of the circumplex and the
fact that real-world perceptual distributions have been consistently
observed to not be strictly symmetric.

The skew-normal distribution is defined by three parameters: location
(\(\mu\)), scale (\(\sigma\)), and shape (\(\alpha\)). The location
parameter defines the centre of the distribution, the scale parameter
defines the spread of the distribution and the shape parameter defines
the skew of the distribution. The one-dimensional skew-normal
distribution is defined as \citet{Azzalini1996Multivariate}:

\[
\phi(z; \alpha) = 2 \phi(z) \Phi(\alpha z) \quad \text{for} \quad z \in \mathbb{R}
\]

where \(\phi\) and \(\Phi\) are the standard normal probability density
function and distribution function, respectively, and \(\alpha\) is a
shape variable which regulates the skewness. The distribution reduces to
a standard normal density when \(\alpha = 0\). The bivariate skew-normal
distribution extends this concept to two dimensions, allowing for the
modelling of asymmetric and skewed distributions in a two-dimensional
space such as the soundscape circumplex. The multivariate skew-normal
distribution including scale and location parameters is given by
combining the normal density and distribution functions
\citep{Azzalini1999Statistical}:

\[
Y = 2 \phi_k (y-\xi; \Omega) \Phi\{\alpha^T\omega^{-1}(y-\xi)\} 
\]

where \(\phi_k\) is the \emph{k}-dimensional normal density with
location \(\xi\), shape \(\alpha\), and covariance matrix \(\Omega\).
\(\Phi \{ \dot \}\) is the normal distribution function and \(\alpha\)
is a \emph{k}-dimensional shape vector. When \(\alpha = 0\), \(Y\)
reduces to the standard multivariate normal \(N_k(\xi, \Omega)\)
density. A circumplex distribution can therefore be parameterised with a
2x2 covariance matrix \(\Omega\), a 2x1 location vector \(\xi\), and a
2x1 shape vector \(\alpha\), written as:

\[
Y \sim SN (\xi, \Omega, \alpha)
\]

By fitting a skew-normal distribution to the soundscape perception
responses, it becomes possible to accurately capture the asymmetry and
skewness of the distribution, enabling a more precise calculation of the
SPI. A bivariate skew-normal distribution can be summarised as a set of
these three parameters. Once parameterised, the distribution can then be
sampled from to generate a synthetic distribution of soundscape
perception responses.

\paragraph{Direct and Centred
parameters}\label{direct-and-centred-parameters}

\section{Defining the SPI Framework}\label{defining-the-spi-framework}

The Soundscape Perception Indices (SPI) framework is centred around the
concept of quantifying the distance between a test distribution of
interest and the desired target distribution. Its goal is to determine
whether a soundscape - whether it be a real-world location, a proposed
design, or a hypothetical scenario - aligns with the desired perception
of that soundscape. This is achieved by first defining the target
distribution, which could represent what is considered to be the `ideal'
soundscape perception for a given context or application. The test
distribution is then compared to the target distribution using a
distance metric, which quantifies the deviation between the two
distributions. The resulting distance value serves as the basis for
calculating the SPI, with smaller distances indicating a closer
alignment between the perceived soundscape and the target soundscape
perception.

Although it is expected that the target distribution would usually
represent the ideal or goal soundscape perception, it is also possible
to define target distributions that represent undesirable or suboptimal
soundscape perceptions. For instance, in a soundscape mapping context,
it may be beneficial to map and identify chaotic soundscapes across a
city in order to better target areas for soundscape interventions. In
this case, the target distribution would be set in the chaotic quadrant
and a higher SPI would indicate a closer alignment with the target
distribution. This flexibility allows the SPI to be applied to a wide
range of contexts and applications, enabling the quantification and
comparison of soundscape quality across diverse scenarios.

An SPI value therefore does not represent a `good' or `bad' soundscape,
but rather a measure of how closely the perceived soundscape aligns with
the desired target soundscape perception. This approach allows for the
development of bespoke indices tailored to specific design goals and
objectives, while also enabling cross-comparisons and benchmarking
against empirically-defined soundscape archetypes.

\subsection{Defining a Target}\label{defining-a-target}

As introduced in Section~\ref{sec-circumplex-distribution}, circumplex
data follows a bivariate skew-normal distribution which can be
parameterised with a set of direct parameters (dp). We therefore define
a target distribution as a set of these parameters, which can then be
used to generate a synthetic distribution of soundscape perception
responses. Three example targets are given below along with their
\(dp_{target}\):

\subsection{Distance Metric}\label{distance-metric}

Central to the SPI framework is the concept of a distance metric, which
quantifies the deviation of a given soundscape from a desired target
soundscape. This distance metric serves as the basis for calculating the
SPI value, with smaller distances indicating a closer alignment between
the perceived soundscape and the target soundscape perception.

Various distance metrics could be employed, ranging from a simple
Euclidean distance

It would be possible to define a single target point, rather than an
entire target distribution and assess the test distribution's distance
from that point using an \(R^2\) based on a euclidian distance. However,
as noted in Mitchell 2022, it is important to also consider the spread
of the distribution. As a key aspect of the sounds, the collective
perception. Of a soundscape.

\emph{Discuss different options of distance metrics and approaches}

Essentially, we approaching this as a problem of (dis)similarity between
soundscapes. The distance metric is then proposed to assess how similar
two any given soundscapes distributions are within the circumplex. Taken
to the extreme, two perfectly matching distributions in the soundscape
circumplex would return a 100\% SPI value, while two completely
dissimilar distributions would return a 0\% SPI value. In practical
terms, for the former, this will never be achieved in real world
scenarios; for the latter, it is also difficult to estimate how low the
SPI value could actually go, and it should be considered that the
distance may happen in different directions within the circumplex space.
For instance, if a distribution for a vibrant soundscape was taken as a
reference, a compared soundscape distribution may exhibit low SPI values
for being located in the calm, OR monotonous, OR chaotic regions of the
model.

\subsection{Targets}\label{targets}

The SPI framework introduces two distinct types of targets: bespoke
targets and archetypal targets, each serving a unique purpose in the
index development process.

\subsubsection{Bespoke Targets}\label{bespoke-targets}

Bespoke targets are tailor-made for specific projects, reflecting the
desired soundscape perception for a particular application. These
targets can be defined by stakeholders, designers, policymakers, or
decision-makers based on their unique requirements, objectives, and
constraints. This flexibility allows the SPI for a specific project to
be tailored to the desire of the stakeholders for how that specific
soundscape should function. It can also provide a consistent and
quantifiable baseline for scenarios like a soundscape design contest
wherein a target is specified and provided to all participants in the
contest and the winning proposal is the design with the highest SPI
score when assessed against that target.

\subsubsection{Archetypal Targets}\label{archetypal-targets}

In contrast to bespoke targets, archetypal targets represent
generalized, widely recognized soundscape archetypes which transcend
specific applications or projects. These archetypes serve as reference
points and enable comparisons across different domains and use cases.
\textbf{\emph{By providing a framework for these archetypes to be
defined, they can be\ldots{}}}

Additionally, archetypal SPIs can be composed of multiple targets.

\subsection{Data Source}\label{data-source}

The SPI framework is designed to accommodate a wide range of data
sources, including both objective measurements and subjective
evaluations. This flexibility enables the framework to be applied to
diverse contexts and applications, ranging from urban soundscapes to
natural environments, public spaces, and indoor settings.

\begin{longtable}[]{@{}llllllllll@{}}
\toprule\noalign{}
& count & ISOPleasant & ISOEventful & pleasant & eventful & vibrant &
chaotic & monotonous & calm \\
\midrule\noalign{}
\endhead
\bottomrule\noalign{}
\endlastfoot
CarloV & 116 & 0.575 & 0.067 & 0.957 & 0.517 & 0.474 & 0.043 & 0.000 &
0.483 \\
SanMarco & 95 & 0.284 & 0.450 & 0.811 & 0.958 & 0.768 & 0.189 & 0.000 &
0.042 \\
PlazaBibRambla & 18 & 0.492 & 0.016 & 0.944 & 0.611 & 0.556 & 0.056 &
0.000 & 0.389 \\
CamdenTown & 105 & -0.022 & 0.408 & 0.467 & 0.914 & 0.410 & 0.505 &
0.029 & 0.057 \\
EustonTap & 96 & -0.118 & 0.237 & 0.312 & 0.771 & 0.240 & 0.531 & 0.156
& 0.073 \\
Noorderplantsoen & 97 & 0.559 & 0.467 & 0.969 & 0.979 & 0.948 & 0.031 &
0.000 & 0.021 \\
MarchmontGarden & 104 & 0.397 & 0.069 & 0.769 & 0.587 & 0.452 & 0.135 &
0.096 & 0.317 \\
MonumentoGaribaldi & 32 & 0.561 & 0.109 & 1.000 & 0.625 & 0.625 & 0.000
& 0.000 & 0.375 \\
TateModern & 152 & 0.467 & 0.312 & 0.928 & 0.862 & 0.789 & 0.072 & 0.000
& 0.138 \\
PancrasLock & 93 & 0.361 & 0.177 & 0.796 & 0.731 & 0.548 & 0.183 & 0.022
& 0.247 \\
TorringtonSq & 113 & 0.179 & 0.273 & 0.681 & 0.796 & 0.540 & 0.257 &
0.062 & 0.142 \\
RegentsParkFields & 107 & 0.709 & 0.043 & 1.000 & 0.570 & 0.570 & 0.000
& 0.000 & 0.430 \\
RegentsParkJapan & 89 & 0.783 & 0.137 & 0.989 & 0.719 & 0.708 & 0.011 &
0.000 & 0.281 \\
RussellSq & 145 & 0.585 & 0.169 & 0.952 & 0.703 & 0.662 & 0.041 & 0.007
& 0.290 \\
StPaulsCross & 66 & 0.454 & 0.242 & 0.909 & 0.773 & 0.712 & 0.061 &
0.030 & 0.197 \\
StPaulsRow & 72 & 0.332 & 0.200 & 0.833 & 0.750 & 0.625 & 0.125 & 0.042
& 0.208 \\
CampoPrincipe & 110 & 0.523 & -0.046 & 0.945 & 0.473 & 0.427 & 0.045 &
0.009 & 0.518 \\
MiradorSanNicolas & 28 & 0.387 & 0.146 & 0.964 & 0.679 & 0.643 & 0.036 &
0.000 & 0.321 \\
\end{longtable}

\section{Applying a Bespoke SPI}\label{applying-a-bespoke-spi}

\section{Case Study - Defining an Archetypal SPI for space
typologies}\label{case-study---defining-an-archetypal-spi-for-space-typologies}

To demonstrate the practical implementation of the SPI framework and
provide an example of empirically-defined targets, a case study focused
on defining a typology-based SPI for public spaces is presented. This
case study utilizes data from the International Soundscape Database
(ISD) \citep{Mitchell2021International}, a comprehensive collection of
soundscape recordings and associated listener evaluations gathered under
the SSID Protocol \citep{Mitchell2020Soundscape}. The SSI Protocol was
specifically designed to capture the multi-dimensional nature of
soundscape perception, employed a rigorous methodology for collecting
and analysing data from diverse public spaces according to the
standardized methods in ISO 12913-2 \citep{ISO12913Part2}.

\subsection{Space Typologies}\label{space-typologies}

The case study focuses on defining an archetypal SPI for public spaces,
with a particular emphasis on space typologies. The concept of space
typologies is rooted in the idea that different types of public spaces,
such as parks, squares, streets, and plazas, exhibit distinct acoustic
characteristics and elicit unique perceptions from their users. By
defining archetypal SPIs for these space typologies, it becomes possible
to establish a standardized framework for evaluating and comparing
public spaces based on their soundscape quality.

The ISD encompasses a diverse range of public space typologies,
including urban parks, city squares, public walkways, and busy streets.
These typologies serve as the basis for defining archetypal targets and
calculating the corresponding SPIs.

\subsection{\texorpdfstring{Defining
\(SPI_{type}\)}{Defining SPI\_\{type\}}}\label{defining-spi_type}

Using the soundscape circumplex model and the perceptual data from the
ISD, the process of defining the \(SPI_{type}\) for each space typology
involves the following steps:

\begin{enumerate}
\def\labelenumi{\arabic{enumi}.}
\tightlist
\item
  Identifying Archetypal Targets: Based on the available data \ldots{}
  target soundscapes are defined for each space typology, representing
  the `ideal' soundscape perception for that particular type of public
  space.
\item
  Calculated \(SPI_{type}\) for each test location: Using the procedure
  given above, the circumplex distribution of each test location is
  compared against the target distribution for its respective space
  typology.
\end{enumerate}

The resulting \(SPI_{type}\) values provide a quantitative measure of
soundscape quality for each space typology, enabling comparisons and
benchmarking across different public spaces. By comparing each test
soundscape against the appropriate target for its typology, the SPI is
able to account for the different contexts and purposes of the
typologies. By using a consistent scoring methodology, SPI then allows
these scores to be combined and considered together, as a single
\(SPI_{type}\) score.

\section{Discussion}\label{discussion}

The development of bespoke and archetypal context-dependent Soundscape
Perception Indices (SPIs) represents a significant step towards enabling
more comprehensive and effective applications of the soundscape
approach. By providing a unified framework for defining these indices,
the potential for quantifying and comparing soundscape quality across
diverse contexts and applications is unlocked, while still ensuring that
the multi-dimensional and context-driven aspects of soundscape quality
are considered.

The proposed framework offers several key advantages. First, it
acknowledges the inherent context-dependent nature of soundscape
perception, allowing for the creation of indices tailored to specific
use cases or design goals through the use of bespoke targets. This
flexibility ensures that the resulting SPIs accurately capture the
desired soundscape perception for the given application, enabling
targeted interventions and optimisations.

Second, the inclusion of archetypal targets facilitates
cross-comparisons and benchmarking, enabling a common language and
understanding of soundscape quality across different domains. By
calculating the distance between a given soundscape and these widely
recognized archetypes, stakeholders can identify areas for improvement
and prioritize interventions accordingly, aligning their efforts with
collectively recognized standards of desirable or undesirable
soundscapes.

The case study presented in this article, focusing on the development of
a typology-based SPI for public spaces, demonstrates the practical
applicability of the framework. By leveraging data from the
International Soundscape Database (ISD) and the SSID Protocol,
archetypal targets for various space typologies were defined, and the
corresponding \(SPI_{type}\) values were calculated. These indices
provide a quantitative measure of soundscape quality for each typology,
enabling comparisons and informing decision-making processes related to
the management and improvement of public spaces.

As stated in \#sec-intro \ldots{}

\citep[Fig.6]{Kogan2018Green}, in fact displays a startlingly similar
concept, showing the locations of the three categories of traffic noise
dominance (`traffic noise', `balanced', and `natural') plotted in the
circumplex perceptual model. It can be clearly seen in this plot that
the GSI categories create their own clusters within the circumplex.

\section{Conclusion}\label{conclusion}

The introduction of bespoke and archetypal context-dependent Soundscape
Perception Indices (SPIs) represents a significant advancement in the
field of soundscape research and application. By providing a unified
framework for defining these indices, a more comprehensive and efficient
approach to quantifying and comparing soundscape quality across diverse
contexts is enabled.

The proposed framework addresses the existing gap in quantifying
multi-dimensional soundscape perception, facilitating a broader
application of the soundscape approach in areas such as urban planning,
environmental management, acoustic design, and policy development.
Through the creation of bespoke indices tailored to specific design
goals and the utilization of archetypal targets for benchmarking, this
framework empowers stakeholders and decision-makers to make informed
choices and prioritize soundscape improvements aligned with their unique
objectives and constraints.

Furthermore, the grounding of the SPI framework in the soundscape
circumplex model ensures a robust theoretical foundation, capturing the
multi-dimensional nature of soundscape perception. The use of a distance
metric enables quantitative assessments and comparisons, fostering a
common language and understanding of soundscape quality across different
domains. This shared understanding facilitates knowledge exchange,
collaborative efforts, and the development of best practices within the
field.

The case study presented in this article, focused on defining a
typology-based SPI for public spaces, demonstrates the practical
applicability of the framework and highlights its potential for enabling
more effective and context-sensitive soundscape management strategies.
By leveraging data from the International Soundscape Database (ISD) and
the SSID Protocol, archetypal targets for various public space
typologies were defined, and the corresponding \(SPI_{type}\) values
were calculated, providing a quantitative measure of soundscape quality
that can inform decision-making processes and guide interventions.

As the SPI framework continues to be explored and refined, future
research should focus on validating and expanding the range of
archetypal targets, as well as investigating the potential for
incorporating additional dimensions and factors that influence
soundscape perception. The integration of emerging technologies, such as
virtual and augmented reality, may also provide new avenues for
immersive soundscape evaluation and index development.

Additionally, the application of the framework in diverse real-world
scenarios, ranging from urban planning and environmental management to
acoustic design and policy development, will provide valuable insights
and contribute to the ongoing refinement and adaptation of the SPI
framework. Collaboration with stakeholders, end-users, and experts from
various domains will be crucial in ensuring the framework's relevance
and applicability across a wide range of contexts.

Furthermore, the development of standardized data collection protocols
and the establishment of comprehensive soundscape databases will be
essential for the widespread adoption and effective implementation of
the SPI framework. Initiatives focused on promoting data sharing,
interoperability, and open access to soundscape data can significantly
facilitate the creation and validation of new indices, fostering a more
collaborative and data-driven approach to soundscape research and
management.

Ultimately, the introduction of bespoke and archetypal context-dependent
Soundscape Perception Indices represents a significant stride towards a
more holistic and nuanced understanding of our acoustic environments,
paving the way for more informed decision-making and enhancing the
overall quality of life in our built and natural environments. By
empowering stakeholders with the ability to quantify and compare
soundscape quality, new avenues are unlocked for targeted interventions,
strategic planning, and the creation of soundscapes that are not only
acoustically optimal but also deeply resonant with the diverse needs and
perceptions of individuals and communities.


  \bibliography{FellowshipRefs-biblatex.bib}


\end{document}
