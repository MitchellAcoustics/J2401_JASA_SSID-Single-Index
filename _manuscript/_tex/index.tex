% Options for packages loaded elsewhere
\PassOptionsToPackage{unicode}{hyperref}
\PassOptionsToPackage{hyphens}{url}
\PassOptionsToPackage{dvipsnames,svgnames,x11names}{xcolor}
%
\documentclass[
  authoryear,
  preprint,
  3p]{elsarticle}

\usepackage{amsmath,amssymb}
\usepackage{iftex}
\ifPDFTeX
  \usepackage[T1]{fontenc}
  \usepackage[utf8]{inputenc}
  \usepackage{textcomp} % provide euro and other symbols
\else % if luatex or xetex
  \usepackage{unicode-math}
  \defaultfontfeatures{Scale=MatchLowercase}
  \defaultfontfeatures[\rmfamily]{Ligatures=TeX,Scale=1}
\fi
\usepackage{lmodern}
\ifPDFTeX\else  
    % xetex/luatex font selection
\fi
% Use upquote if available, for straight quotes in verbatim environments
\IfFileExists{upquote.sty}{\usepackage{upquote}}{}
\IfFileExists{microtype.sty}{% use microtype if available
  \usepackage[]{microtype}
  \UseMicrotypeSet[protrusion]{basicmath} % disable protrusion for tt fonts
}{}
\makeatletter
\@ifundefined{KOMAClassName}{% if non-KOMA class
  \IfFileExists{parskip.sty}{%
    \usepackage{parskip}
  }{% else
    \setlength{\parindent}{0pt}
    \setlength{\parskip}{6pt plus 2pt minus 1pt}}
}{% if KOMA class
  \KOMAoptions{parskip=half}}
\makeatother
\usepackage{xcolor}
\setlength{\emergencystretch}{3em} % prevent overfull lines
\setcounter{secnumdepth}{5}
% Make \paragraph and \subparagraph free-standing
\ifx\paragraph\undefined\else
  \let\oldparagraph\paragraph
  \renewcommand{\paragraph}[1]{\oldparagraph{#1}\mbox{}}
\fi
\ifx\subparagraph\undefined\else
  \let\oldsubparagraph\subparagraph
  \renewcommand{\subparagraph}[1]{\oldsubparagraph{#1}\mbox{}}
\fi


\providecommand{\tightlist}{%
  \setlength{\itemsep}{0pt}\setlength{\parskip}{0pt}}\usepackage{longtable,booktabs,array}
\usepackage{calc} % for calculating minipage widths
% Correct order of tables after \paragraph or \subparagraph
\usepackage{etoolbox}
\makeatletter
\patchcmd\longtable{\par}{\if@noskipsec\mbox{}\fi\par}{}{}
\makeatother
% Allow footnotes in longtable head/foot
\IfFileExists{footnotehyper.sty}{\usepackage{footnotehyper}}{\usepackage{footnote}}
\makesavenoteenv{longtable}
\usepackage{graphicx}
\makeatletter
\def\maxwidth{\ifdim\Gin@nat@width>\linewidth\linewidth\else\Gin@nat@width\fi}
\def\maxheight{\ifdim\Gin@nat@height>\textheight\textheight\else\Gin@nat@height\fi}
\makeatother
% Scale images if necessary, so that they will not overflow the page
% margins by default, and it is still possible to overwrite the defaults
% using explicit options in \includegraphics[width, height, ...]{}
\setkeys{Gin}{width=\maxwidth,height=\maxheight,keepaspectratio}
% Set default figure placement to htbp
\makeatletter
\def\fps@figure{htbp}
\makeatother

\makeatletter
\@ifpackageloaded{caption}{}{\usepackage{caption}}
\AtBeginDocument{%
\ifdefined\contentsname
  \renewcommand*\contentsname{Table of contents}
\else
  \newcommand\contentsname{Table of contents}
\fi
\ifdefined\listfigurename
  \renewcommand*\listfigurename{List of Figures}
\else
  \newcommand\listfigurename{List of Figures}
\fi
\ifdefined\listtablename
  \renewcommand*\listtablename{List of Tables}
\else
  \newcommand\listtablename{List of Tables}
\fi
\ifdefined\figurename
  \renewcommand*\figurename{Figure}
\else
  \newcommand\figurename{Figure}
\fi
\ifdefined\tablename
  \renewcommand*\tablename{Table}
\else
  \newcommand\tablename{Table}
\fi
}
\@ifpackageloaded{float}{}{\usepackage{float}}
\floatstyle{ruled}
\@ifundefined{c@chapter}{\newfloat{codelisting}{h}{lop}}{\newfloat{codelisting}{h}{lop}[chapter]}
\floatname{codelisting}{Listing}
\newcommand*\listoflistings{\listof{codelisting}{List of Listings}}
\makeatother
\makeatletter
\makeatother
\makeatletter
\@ifpackageloaded{caption}{}{\usepackage{caption}}
\@ifpackageloaded{subcaption}{}{\usepackage{subcaption}}
\makeatother
\journal{Journal of the Acoustical Society of America}
\ifLuaTeX
  \usepackage{selnolig}  % disable illegal ligatures
\fi
\usepackage[]{natbib}
\bibliographystyle{elsarticle-harv}
\usepackage{bookmark}

\IfFileExists{xurl.sty}{\usepackage{xurl}}{} % add URL line breaks if available
\urlstyle{same} % disable monospaced font for URLs
\hypersetup{
  pdftitle={SPI - Defining bespoke and archetypal context-dependent Soundscape Perception Indices},
  pdfauthor={Andrew Mitchell; Francesco Aletta; Tin Oberman; Jian Kang},
  pdfkeywords={keyword1, keyword2},
  colorlinks=true,
  linkcolor={blue},
  filecolor={Maroon},
  citecolor={Blue},
  urlcolor={Blue},
  pdfcreator={LaTeX via pandoc}}

\setlength{\parindent}{6pt}
\begin{document}

\begin{frontmatter}
\title{SPI - Defining bespoke and archetypal context-dependent
Soundscape Perception Indices}
\author[1]{Andrew Mitchell%
\corref{cor1}%
}
 \ead{andrew.mitchell.18@ucl.ac.uk} 
\author[1]{Francesco Aletta%
%
}
 \ead{f.aletta@ucl.ac.uk} 
\author[1]{Tin Oberman%
%
}
 \ead{t.oberman@ucl.ac.uk} 
\author[]{Jian Kang%
%
}
 \ead{j.kang@ucl.ac.uk} 

\affiliation[1]{organization={University College London, Institute for
Environmental Design \& Engineering},,postcodesep={}}

\cortext[cor1]{Corresponding author}




        
\begin{abstract}
The soundscape approach provides a basis for considering the holistic
perception of sound environments, in context. While steady advancements
have been made in methods for assessment and analysis, a gap exists for
comparing soundscapes and quantifying improvements in the
multi-dimensional perception of a soundscape. To this end, there is a
need for the creation of single value indices to compare soundscape
quality which incorporate context, aural diversity, and specific design
goals for a given application. Just as a variety of decibel-based
indices have been developed for various purposes (e.g.~\(L_{Aeq}\),
\(L_{Ceq}\), \(L_{90}\), \(L{den}\), etc.), the soundscape approach
requires the ability to create novel indices for different uses, but
which share a common language and understanding. We therefore propose a
unified framework for creating both bespoke and standardised single
index measures of soundscape perception based on the soundscape
circumplex model, allowing for new metrics to be defined in the future.
The implementation of this framework is demonstrated through the
creation of a public spaced typology-based index using data collected
under the SSID Protocol, which was designed specifically for the purpose
of defining soundscape indices. Indices developed under this framework
can enable a broader and more efficient application of the soundscape
approach.
\end{abstract}





\begin{keyword}
    keyword1 \sep 
    keyword2
\end{keyword}
\end{frontmatter}
    
\section{Introduction}\label{introduction}

The EU Green Paper on Future Noise Policy indicates that 80 million EU
citizens are suffering from unacceptable environmental noise levels,
according to the WHO recommendation \citep{Berglund1999Guidelines} and
the social cost of transport noise is 0.2-2\% of total GDP. The
publication of the EU Directive Relating to the Assessment and
Management of Environmental Noise (END)
\citep{EuropeanUnion2002Directive} in 2002 has led to major actions
across Europe, with reducing noise levels as the focus, for which
billions of Euros are being spent. However, it is widely recognised that
solely reducing sound level is not always feasible or cost-effective,
and more importantly, with only \textasciitilde30\% of environmental
noise annoyance depending on facets of parameters such as acoustic
energy \citep{Guski1997Psychological}, sound level reduction will not
necessarily lead to improved quality of life.

Soundscape design, separate from noise control engineering, is about the
relationships between human physiology, perception, the sound
environment, and its social/cultural context \citep{Kang2006Urban}.
Soundscape research represents a paradigm shift in that it combines
physical, social, and psychological approaches and considers
environmental sounds as a `resource' rather than `waste'
\citep{Kang2016Soundscape} relating to perceptual constructs rather than
just physical phenomena. However, the current research is still at the
stage of describing and identifying the problems and tends to be
fragmented and focussed on only special cases e.g.~subjective
evaluations of soundscapes for residential areas
\citep{SchulteFortkamp2013Introduction}. In the movement from noise
control to soundscape creation \citep{Aletta2015Soundscape}, a vital
step is the standardisation of methods to assess soundscape quality.

A common aim for implementing soundscape assessment in practice is to
compare the quality of different soundscapes. Often (but not always) the
goal is to identify a `good' soundscape compared to a `bad' soundscape.
However, this presents several challenges:

\begin{itemize}
\tightlist
\item
  What makes a soundscape good or bad is highly contextual;
\item
  On what metric should the quality rating be based?
\item
  How can we deal with different requirements and definitions of how a
  soundscape should be perceived?
\end{itemize}

In many cases, the ultimate aim is to be able to rank soundscapes based
on their quality. However, any ranking metric should be flexible and be
able to handle a variety of contexts and definitions of what a `good'
soundscape is for a given purpose. To address this, we will propose the
Soundscape Perception Index (SPI) framework, a flexible method for
defining single value indices of soundscape quality based on
distributions within the Soundscape Circumplex Model (SCM)
\citep[\citet{Axelsson2012Swedish},
\citet{Mitchell2022How}]{Axelsson2010principal}.

The primary motivation behind the development of the Soundscape
Perception Indices (SPIs) framework stems from the need to address the
existing gap in quantifying and comparing soundscape quality across
diverse contexts and applications. By creating a unified framework for
defining these indices, the aim is to facilitate a broader and more
efficient application of the soundscape approach in various domains,
such as urban planning, environmental management, acoustic design, and
policy development.

The overarching aim of this framework is to empower stakeholders,
decision-makers, and researchers with the ability to create tailored
indices that align with their specific objectives and design goals,
while simultaneously enabling cross-comparisons and benchmarking against
empirically-defined soundscape archetypes. This dual approach not only
acknowledges the context-dependent nature of soundscape perception but
also fosters a common language and understanding, facilitating knowledge
sharing and collaborative efforts within the field. This paper will
demonstrate the SPI framework and test whether it is capable of both
scoring soundscape quality and generating consistent rankings of
soundscapes across different contexts.

\section{Background}\label{background}

In \citet{Aletta2016Soundscape}, the authors defined a framework for
categorising the components of a soundscape assessment. They define
three aspects: soundscape descriptors, soundscape indicators, and
soundscape indices. Soundscape descriptors are defined as `measures of
how people perceive the acoustic environment' and soundscape indicators
as `measures used to predict the value of a soundscape descriptor'.
Indices, the primary focus of this article, are single numerical values
that combine multiple indicators or descriptors to provide a
comprehensive representation of the overall soundscape perception and
allow for comparison between soundscapes. These indices serve as
powerful tools for quantifying and comparing soundscapes, enabling
decision-makers and stakeholders to assess the impact of interventions,
monitor changes over time, and prioritize areas for
improvement\citep{Kang2019Towards}.

The earliest and most commonly used scientific index measuring sound
level is the Decibel (dB). To represent the overall level of sound with
a single value on one scale, as the Decibel index does, is often
desirable. For this purpose, a number of different values representing
sounds at various frequencies must be combined. Several frequency
weighting networks have been developed since the 1930s, considering
typical human responses to sound based on equal-loudness-level contours
\citep{Fletcher1933Loudness} and, among them, the A-weighting network,
with resultant decibel values called dBA, has been commonly used in
almost all the national/international regulations
\citep{Kryter1970Effects}. However, there have been numerous criticisms
on its effectiveness \citep{Parmanen2007weighted} as the correlations
between dBA and perceived sound quality (e.g.~noise annoyance) are often
low \citep{Hellman1987Why}.

Another set of indices is psychoacoustic magnitudes, including loudness,
fluctuation strength or roughness, sharpness, and pitch strength,
development with sound quality studies of industrial products since the
1980's \citep{Zwicker2007Psychoacoustics}. These emerged when it was
conceived that acoustic emissions can be characterised beyond just sound
level \citep{Blauert1997Sound}. But while psychoacoustic magnitudes have
proven to be successful for the assessment of product sound quality, in
the field of environmental acoustics, their applicability has been
limited \citep{Fastl2006Psychoacoustic}, since a significant feature of
environmental acoustics is that there are multiple/dynamic sound
sources. Additionally, while pyschoacoustic magnitudes incorporate
perceptual aspects, both dB based and pyschoacoustic indicies are
ultimately describing the acoustic signal and not the soundscape
perception and may therefore be more accurately described as indicators
rather than soundscape indices \citep{Mitchell2023conceptual}.

When applied to urban sound and specifically to noise pollution, the
soundscape approach introduces three key considerations beyond
traditional noise control methods:

\begin{enumerate}
\def\labelenumi{\arabic{enumi}.}
\tightlist
\item
  considering all aspects of the environment which may influence
  perception, not just the sound level and spectral content;
\item
  an increased and integrated consideration of the varying impacts which
  different sound sources and sonic characteristics have on perception;
  and
\item
  a consideration of both the positive and negative dimensions of
  soundscape perception.
\end{enumerate}

This approach can enable better outcomes by identifying positive
soundscapes (in line with the END's mandate to `preserve environmental
noise quality where it is good' \citep{EuropeanUnion2002Directive}),
better identify specific sources of noise which impact soundscape
quality and pinpoint the characteristics which may need to be decreased,
and illuminate alternative methods which could be introduced to improve
a soundscape where a reduction of noise is impractical
\citep{Fiebig2018Does, Kang2018Impact}. These can all lead to more
opportunities to truly improve a space by identifying the causes of
positive soundscapes, while also potentially decreasing the costs of
noise mitigation by offering more targeted techniques and alternative
approaches.

The traditional focus on noise levels alone fails to capture the
complexity of soundscape perception, which encompasses a multitude of
factors beyond mere sound pressure levels. Factors such as the presence
of natural or human-made sounds, their temporal patterns, and the
overall contextual meaning ascribed to these sounds all contribute to
the holistic perception of a soundscape. Consequently, there is a
pressing need for the development of robust indices that can encapsulate
this multi-dimensional nature of soundscape perception, enabling
comparative evaluations and informing targeted interventions to enhance
the overall quality of acoustic environments \citep{Chen2023Developing}.

\subsection{Existing `Soundscape
Indices'}\label{existing-soundscape-indices}

While the field of soundscape research has witnessed substantial
progress, the development of standardized indices for evaluating and
comparing soundscapes across diverse contexts has been relatively
limited. Existing indices can be broadly seen as arising from two
domains: soundscape ecology and soundscape perception.

\subsubsection{Soundscape Ecology and
Bioacoustics}\label{soundscape-ecology-and-bioacoustics}

Within the realm of soundscape ecology, indices such as the Acoustic
Diversity Index (ADI) and Frequency-dependenty Acoustic Diversity Index
(FADI) \citep{Xu2023frequency} have been developed to quantify the
diversity and complexity of acoustic signals within a given soundscape.
Similar indices (e.g.~ADI, NDSI, ACI) have also been developed to
analyse the acoustic signal of complex acoustic environments and
indicate the richness and diversity of biophonic (natural) and
anthrophonic (human-made) sound sources. However, while these indices
contribute valuable insights into the ecological aspects of soundscapes,
they do not directly address the perceptual dimensions that are central
to the soundscape approach \citep{SchulteFortkamp2023Soundscapes}. The
multi-dimensional nature of soundscape perception, encompassing factors
such as pleasantness, eventfulness, and familiarity, necessitates a more
comprehensive and context-sensitive approach.

\subsubsection{Soundscape Perception}\label{soundscape-perception}

In the domain of soundscape perception, the Green Soundscape Index (GSI)
\citep{Kogan2018Green} has emerged as a notable attempt to quantify the
perceived quality of soundscapes, particularly in urban environments.
This index incorporates factors such as the presence and levels of
natural sounds, human-made sounds, and their respective contributions to
the overall soundscape perception.

The GSI is the ratio of the perceived extent of natural sounds (PNS) to
the perceived extent of traffic noise (PTN):

\[
GSI = \frac{<PNS>}{<PTN>}
\]

The GSI is noted to range between 1/5 and 5, with several ranges of
values given which correspond to general categories of the perceived
dominance of traffic noise.

While GSI represents a commendable effort to bridge the gap between
objective measurements and subjective perceptions, it remains limited in
its ability to capture the full complexity of soundscape perception
across diverse contexts. The intricate interplay between various sound
sources, their temporal patterns, and the specific context in which they
are experienced necessitates a more flexible and adaptable approach to
index development.

The Soundscape Perception Index framework presented in this paper
differs from these existing indices in two key ways. Firstly, it is not
an analysis of an acoustic signal but rather is an index of perception
based on soundscape descriptors. Secondly, it does not represent a
single target in a particular context, but is a generalisable,
extensible, and adaptable framework for scoring soundscapes against any
goal defined by the user. The remainder of the paper will introduce and
demonstrate this framework, providing a case study of defining an
appropriate target.

\section{Methodology}\label{sec-method}

The index framework, `the Soundscape Perception Indices (SPI)'
introduced in this paper is defined here as the agreement between an
observed or modelled soundscape perception distribution and a target
soundscape perception distribution. Its goal is to determine whether a
soundscape - whether it be a real-world location, a proposed design, or
a hypothetical scenario - aligns with the desired perception of that
soundscape. This is achieved by first defining the target distribution,
which could represent what is considered to be the `ideal' soundscape
perception for a given context or application. The test distribution is
then compared to the target distribution using a distance metric, which
quantifies the deviation between the two distributions. The resulting
distance value serves as the basis for calculating the SPI, with smaller
distances indicating a closer alignment between the perceived soundscape
and the target soundscape perception.

We refer to this as an index framework rather than a single index, as
the SPI can be tailored to specific contexts and applications by
defining a range of target distributions. A single index is thus created
for each target distribution. An SPI value therefore does not represent
a `good' or `bad' soundscape, but rather a measure of how closely the
perceived soundscape aligns with the desired target soundscape
perception. This approach allows for the development of bespoke indices
tailored to specific design goals and objectives, while also enabling
cross-comparisons and benchmarking against empirically-defined
soundscape archetypes.

SPI is grounded in the soundscape circumplex model (SCM)
\citep{Axelsson2010principal, Axelsson2012Swedish}, a robust theoretical
foundation for understanding and representing the multi-dimensional
nature of soundscape perception. The reason for grounding the SPI in the
soundscape circumplex is that we have observed this model (and its
corresponding PAQs) to become the most prevalent assessment model in
soundscape literature \citep{Aletta2023Adoption}.

The SCM is built on a series of descriptors referred to as the Perceived
Affective Quality (PAQ), proposed by \citep{Axelsson2010principal}.
These PAQs are based on the pleasantness-activity paradigm present in
research on emotions and environmental psychology, in particular
Russell's circumplex model of affect \citep{Russell1980circumplex}. As
summarised by Axelsson: ``Russell's model identifies two dimensions
related to the perceived pleasantness of environments and how activating
or arousing the environment is.''

One benefit of the circumplex model is that, as a whole, it encapsulates
several of the other proposed soundscape descriptors - in particular,
annoyance, pleasantness, tranquility, and possibly restorativeness
\citep{Aletta2016Soundscape}. According to \citet{Axelsson2015How}, the
two-dimensional circumplex model of perceived affective quality provides
the most comprehensive information for soundscape assessment. It is also
possible that the overall soundscape quality could itself be derived
from the pleasant-eventful scores derived for a soundscape. The
circumplex also lends itself well to questionnaire-based methods of data
collection, as proposed in \citet{ISO12913Part2}. In contrast to methods
such as soundwalks, interviews, and lab experiments, in-situ
questionnaires are able to provide the quality and amount of data which
is necessary for statistical modelling. Combined, these factors make the
circumplex most appropriate for a single index as it provides a
comprehensive summary of soundscape perception.

There are four steps involved in calculating the SPI, as shown in
Figure~\ref{fig-bespoke-spi}:

\begin{enumerate}
\def\labelenumi{\arabic{enumi}.}
\tightlist
\item
  Define and parameterise the target circumplex distribution;
\item
  Sample the target distribution and prepare the test distribution;
\item
  Compare test and target distributions using the distance metric
  (2-dimensional Kolmogorov-Smirnov distance);
\item
  Calculate \(SPI = 100 * (1 - KS)\).
\end{enumerate}

\begin{figure}

\centering{

\includegraphics[width=0.8\textwidth,height=\textheight]{SPI-steps2.png}

}

\caption{\label{fig-bespoke-spi}Steps for calculating the SPI.}

\end{figure}%

These steps and their required background are discussed in detail in the
following sections. Section~\ref{sec-targets} will then present
strategies for defining targets and their applications.

Throughout this paper, we use the data contained in the International
Soundscape Database (ISD) \citep{Mitchell2024International}, which
includes 1300+ individual responses on the PAQ scales collected across
13 locations in London and Venice, according to the SSID Protocol
\citet{Mitchell2020Soundscape}.

\subsection{Define and Parameterise a Soundscape Circumplex
Distribution}\label{sec-circumplex-distribution}

To move the 8-item PAQ responses into the 2-dimensional circumplex
space, we use the projection method first presented in ISO 12913-3:2018.
This projection method and its associated formulae were recently updated
further in \citet{Mitchell2023Testing} to include a correction for the
language in which the survey was conducted. \citet{Mitchell2023Testing}
also provides adjusted angles for translations of the circumplex
attributes to be used in calculating the \(P_{ISO}\) and \(E_{ISO}\)
coordinates.

Once the individual perceptual responses are projected into the
circumplex space, the resulting data for each location is treated as a
circumplex distribution. There are several advancements in considering
circumplex distributions compared to the discussions originally given in
\citet{Mitchell2022How}, which are necessary for SPI. Before exploring
the SPI method and target setting more specifically, we will first
address these developments.

The circumplex is defined by two axes: \(P_{ISO}\) and \(E_{ISO}\),
which are limited to the range \([-1, +1]\). Typically, data in the
soundscape circumplex is treated as a combination of two independent
normal distributions, one for each axis
\citep{Mitchell2022How, Ooi2022Probably}. In some applications this
approach is sufficient for capturing the distribution of soundscape
perception, however defining a target distribution for SPI requires a
more precise approach. The independent normal distribution approach
relies on three key assumptions:

\begin{enumerate}
\def\labelenumi{\arabic{enumi}.}
\tightlist
\item
  The two axes are normally distributed.
\item
  The two axes are independent of each other.
\item
  The two axes are symmetrically distributed.
\end{enumerate}

While the first assumption is generally valid, the second and third
assumptions are not always met in practice. In particular, the
distribution of soundscape perception responses in the circumplex is
often characterised by a high degree of skewness, which can lead to
inaccuracies in the calculation of the SPI. Soundscape circumplex
distributions are most appropriately described as a bivariate
skew-normal distribution \citep{Azzalini2005Skew} which accurately
reflects the relationship between the two dimensions of the circumplex
and the fact that real-world perceptual distributions have been
consistently observed to not be strictly symmetric.

The skew-normal distribution is defined by three parameters: location
(\(\mu\)), scale (\(\sigma\)), and shape (\(\alpha\)). The location
parameter defines the centre of the distribution, the scale parameter
defines the spread of the distribution and the shape parameter defines
the skew of the distribution. The one-dimensional skew-normal
distribution is defined as \citep{Azzalini1996Multivariate}:

\[
\phi(z; \alpha) = 2 \phi(z) \Phi(\alpha z) \quad \text{for} \quad z \in \mathbb{R}
\]

where \(\phi\) and \(\Phi\) are the standard normal probability density
function and distribution function, respectively, and \(\alpha\) is a
shape variable which regulates the skewness. The distribution reduces to
a standard normal density when \(\alpha = 0\). The bivariate skew-normal
distribution extends this concept to two dimensions, allowing for the
modelling of asymmetric and skewed distributions in a two-dimensional
space such as the soundscape circumplex. The multivariate skew-normal
(MSN) distribution including scale and location parameters is given by
combining the normal density and distribution functions
\citep{Azzalini1999Statistical}:

\[
Y = 2 \phi_k (y-\xi; \Omega) \Phi\{\alpha^T\omega^{-1}(y-\xi)\} 
\]

where \(\phi_k\) is the \emph{k}-dimensional normal density with
location \(\xi\), shape \(\alpha\), and covariance matrix \(\Omega\).
\(\Phi \{ \dot \}\) is the normal distribution function and \(\alpha\)
is a \emph{k}-dimensional shape vector. When \(\alpha = 0\), \(Y\)
reduces to the standard multivariate normal \(N_k(\xi, \Omega)\)
density. A circumplex distribution can therefore be
parameterised\footnote{It is important to note that the parameters which
  appear in the density expression (\(\xi, \Omega, \alpha\)) are what
  are called `direct parameters' (DP). They directly parameterise an MSN
  density and are typically only estimated by fitting an MSN to a
  sample. The more familiar and interpretable components (mean, standard
  deviation, and skewness) are termed the centred parameters (CP). It is
  possible to move from one parameterization to another, however ``while
  any choice of the DP components is admissible, the same is not true
  for CP''; i.e.~we can always move DP \(\rightarrow\) CP but not always
  CP \(\rightarrow\) DP. In this context, it is most important for
  readers not to confuse the location parameter \(\xi\) with the sample
  mean \(\mu\). A more complete explanation of these parameterizations
  can be found in \citet{Azzalini2016How}} with a 2x2 covariance matrix
\(\Omega\), a 2x1 location vector \(\xi\), and a 2x1 shape vector
\(\alpha\), written as:

\[
Y \sim MSN (\xi, \Omega, \alpha)
\]

By fitting an MSN distribution to empirical soundscape perception
responses, it becomes possible to accurately capture the asymmetry and
skewness of the distribution. A bivariate skew-normal distribution can
be summarised as a set of these three parameters. Once parameterised,
the distribution can then be sampled from to generate a synthetic
distribution of soundscape perception responses.

Soundscape targets can thus be set by defining the desired MSN
distribution. To demonstrate this, we will construct three arbitrary
targets which will be used later to score three SPIs. The parameters
chosen for the example targets are given in
Table~\ref{tbl-target-params}.

\begin{longtable}[]{@{}
  >{\centering\arraybackslash}p{(\columnwidth - 6\tabcolsep) * \real{0.1373}}
  >{\centering\arraybackslash}p{(\columnwidth - 6\tabcolsep) * \real{0.1373}}
  >{\centering\arraybackslash}p{(\columnwidth - 6\tabcolsep) * \real{0.5882}}
  >{\centering\arraybackslash}p{(\columnwidth - 6\tabcolsep) * \real{0.1373}}@{}}
\caption{The MSN direct parameterizations for three arbitrary example
target distributions. \(\text{tgt}_1\) is located in the pleasant half,
with a wide variance, and a positive skew along the pleasantness
axis.}\label{tbl-target-params}\tabularnewline
\toprule\noalign{}
\begin{minipage}[b]{\linewidth}\centering
Target
\end{minipage} & \begin{minipage}[b]{\linewidth}\centering
Location \(\xi\)
\end{minipage} & \begin{minipage}[b]{\linewidth}\centering
Covariance Matrix \(\Omega\)
\end{minipage} & \begin{minipage}[b]{\linewidth}\centering
Shape \(\alpha\)
\end{minipage} \\
\midrule\noalign{}
\endfirsthead
\toprule\noalign{}
\begin{minipage}[b]{\linewidth}\centering
Target
\end{minipage} & \begin{minipage}[b]{\linewidth}\centering
Location \(\xi\)
\end{minipage} & \begin{minipage}[b]{\linewidth}\centering
Covariance Matrix \(\Omega\)
\end{minipage} & \begin{minipage}[b]{\linewidth}\centering
Shape \(\alpha\)
\end{minipage} \\
\midrule\noalign{}
\endhead
\bottomrule\noalign{}
\endlastfoot
\(\text{tgt}_1\) & \([0.5, 0.0]\) &
\(\begin{bmatrix} 0.2 & 0.0 \\ 0.0 & 0.2 \end{bmatrix}\) & \([1, 0]\) \\
\(\text{tgt}_2\) & \([1.0, -0.4]\) &
\(\begin{bmatrix} 0.18 & -0.04 \\ -0.04 & 0.09 \end{bmatrix}\) &
\([-8, 1]\) \\
\(\text{tgt}_3\) & \([0.5, 0.7]\) &
\(\begin{bmatrix} 0.1 & 0.05 \\ 0.05 & 0.1 \end{bmatrix}\) &
\([0, -5]\) \\
\end{longtable}

\subsection{Sample a Target
Distribution}\label{sample-a-target-distribution}

Once the parameters for an MSN are defined (i.e.~the target), the MSN is
then sampled using the \texttt{sn} package \citep{Azzalini2021R} in
\texttt{R} \citep{RCT2018R}. This is to prepare the target distribution
to be compared with the empirical test distribution. Several
restrictions to the possible parameter values apply, most importantly
the covariance matrix \(\Omega\) must be a positive-definite matrix. In
depth discussions of how these parameterizations should be defined and
their restrictions can be found in \citet{Azzalini2016How}.
Figure~\ref{fig-targets} shows the result of sampling (n=1000) the three
example distributions given in Table~\ref{tbl-target-params} and
plotting them as soundscape distributions.

\begin{figure}[H]

\centering{

\includegraphics{index_files/figure-latex/notebooks-SingleIndex-Code-fig-targets-output-1.png}

}

\caption{\label{fig-targets}Example of defining and sampling from three
arbitrary bespoke targets.}

\end{figure}%

\textsubscript{Source:
\href{https://MitchellAcoustics.github.io/J2401_JASA_SSID-Single-Index/notebooks/SingleIndex-Code.ipynb.html\#cell-fig-targets}{SPI
- Defining bespoke and archetypal context-dependent Soundscape
Perception Indices}}

\subsection{Compare the target and test
distributions}\label{compare-the-target-and-test-distributions}

Central to the SPI framework is the concept of a distance metric, which
quantifies the deviation of a given soundscape from a desired target
soundscape. This distance metric serves as the basis for calculating the
SPI value, with smaller distances indicating a closer alignment between
the perceived soundscape and the target soundscape perception. The
distance between the test and target soundscape distributions is
calculated using a two-dimensional Kolmogorov-Smirnov test
\citep{Fasano1987multidimensional}. The KS test is a non-parametric test
of the equality of continuous distributions which is sensitive to both
the location and shape of the distributions
\citep{Chakravati1967Handbook}.

Various other distance metrics were considered when developing the SPI
method. The simplest method is to define a single point target, rather
than a target distribution, and calculate a normalized mean Euclidean
distance between points in the test distribution and the target point.
While this is conceptually simple and requires defining only a single
coordinate point as a target, rather than the MSN parameters described
in Section~\ref{sec-circumplex-distribution}, the shape and spread of a
soundscape distribution is itself an important factor in describing the
collective perception of a soundscape and would not be captured by this
method \citep{Mitchell2022How}.

Essentially, we approach this as a problem of (dis)similarity between
soundscapes. The distance metric is then proposed to assess how similar
any two given soundscapes distributions are within the circumplex. Taken
to the extreme, two perfectly matching distributions in the soundscape
circumplex would return a 100\% SPI value, while two completely
dissimilar distributions would return a 0\% SPI value. In practical
terms, for the former, this will never be achieved in real world
scenarios; for the latter, it is also difficult to estimate how low the
SPI value could actually go, and it should be considered that the
distance may happen in different directions within the circumplex space.
For instance, if a distribution for a vibrant soundscape was taken as a
reference, a compared soundscape distribution may exhibit low SPI values
for being located in the calm, OR monotonous, OR chaotic regions of the
model.

Using the data from one location in the ISD (Piazza San Marco) as the
test distribution, the KS statistic and p-value is calculated for each
of the target distributions defined above, shown in
Table~\ref{tbl-ks-test}.

\begin{longtable}[]{@{}
  >{\raggedright\arraybackslash}p{(\columnwidth - 4\tabcolsep) * \real{0.1528}}
  >{\raggedright\arraybackslash}p{(\columnwidth - 4\tabcolsep) * \real{0.0972}}
  >{\raggedright\arraybackslash}p{(\columnwidth - 4\tabcolsep) * \real{0.1944}}@{}}

\caption{\label{tbl-ks-test}Kolmogorov-Smirnov test comparing the
empirical test distribution (Piazza San Marco) against three soundscape
target distributions.}

\tabularnewline

\toprule\noalign{}
\begin{minipage}[b]{\linewidth}\raggedright
Target
\end{minipage} & \begin{minipage}[b]{\linewidth}\raggedright
D
\end{minipage} & \begin{minipage}[b]{\linewidth}\raggedright
\begin{verbatim}
      p
\end{verbatim}
\end{minipage} \\
\midrule\noalign{}
\endhead
\bottomrule\noalign{}
\endlastfoot
\(tgt_1\) & 0.66 & 8.59797e-25 \\
\(tgt_2\) & 0.84 & 2.11342e-39 \\
\(tgt_3\) & 0.29 & 2.11342e-39 \\

\end{longtable}

\textsubscript{Source:
\href{https://MitchellAcoustics.github.io/J2401_JASA_SSID-Single-Index/notebooks/SingleIndex-Code.ipynb.html\#cell-tbl-ks-test}{SPI
- Defining bespoke and archetypal context-dependent Soundscape
Perception Indices}}

For the 2D KS test, a p-value less than 0.05 indicates that the
empirical distributions are not drawn from the same distribution
function. In this use case, where we never expect the distributions to
be identical and instead only wish to characterize their degree of
(dis)similarity, we discard the p-value and focus only on the test
statistic.

\subsection{Calculate the SPI score}\label{calculate-the-spi-score}

The final step is to convert the KS test statistic into a more
interpretable form to use as a comparison across soundscapes. Since the
KS test statistic is a measure of dissimilarity, we first subtract it
from one to give a measure of similarity between the test distribution
and the target distribution. This is then scaled to produce a score
which ranges from 0 to 100, giving the final SPI formula:

\[
\text{SPI} = 100 * (1 - KS\{\text{MSN}_{test}, \text{MSN}_{tgt}\})
\]

The three SPIs can now be calculated for all of the locations in the
ISD, shown in Table~\ref{tbl-ex-spis}. This produces three separate
rankings of soundscape quality for these locations, depending on which
target is considered the goal.

\begin{longtable}[]{@{}
  >{\raggedleft\arraybackslash}p{(\columnwidth - 6\tabcolsep) * \real{0.1279}}
  >{\raggedright\arraybackslash}p{(\columnwidth - 6\tabcolsep) * \real{0.2907}}
  >{\raggedright\arraybackslash}p{(\columnwidth - 6\tabcolsep) * \real{0.2907}}
  >{\raggedright\arraybackslash}p{(\columnwidth - 6\tabcolsep) * \real{0.2907}}@{}}

\caption{\label{tbl-ex-spis}SPI scores and rankings for the soundscapes
of locations included in the International Soundscape Database (ISD).}

\tabularnewline

\toprule\noalign{}
\begin{minipage}[b]{\linewidth}\raggedleft
Ranking
\end{minipage} & \begin{minipage}[b]{\linewidth}\raggedright
\(SPI_1\)
\end{minipage} & \begin{minipage}[b]{\linewidth}\raggedright
\(SPI_2\)
\end{minipage} & \begin{minipage}[b]{\linewidth}\raggedright
\(SPI_3\)
\end{minipage} \\
\midrule\noalign{}
\endhead
\bottomrule\noalign{}
\endlastfoot
1 & 72 RegentsParkFields & 61 CampoPrincipe & 70 SanMarco \\
2 & 70 CarloV & 52 CarloV & 61 TateModern \\
3 & 65 RegentsParkJapan & 49 PlazaBibRambla & 61 Noorderplantsoen \\
4 & 62 CampoPrincipe & 49 RegentsParkFields & 60 StPaulsCross \\
5 & 62 MarchmontGarden & 44 MonumentoGaribaldi & 54 PancrasLock \\
6 & 61 PlazaBibRambla & 44 MarchmontGarden & 52 TorringtonSq \\
7 & 61 RussellSq & 41 RussellSq & 46 StPaulsRow \\
8 & 61 MonumentoGaribaldi & 40 PancrasLock & 46 RussellSq \\
9 & 59 PancrasLock & 38 RegentsParkJapan & 45 MiradorSanNicolas \\
10 & 52 StPaulsCross & 31 StPaulsCross & 41 CamdenTown \\
11 & 48 TateModern & 31 MiradorSanNicolas & 39 CarloV \\
12 & 47 StPaulsRow & 30 TateModern & 36 MonumentoGaribaldi \\
13 & 42 MiradorSanNicolas & 29 StPaulsRow & 33 MarchmontGarden \\
14 & 40 Noorderplantsoen & 28 TorringtonSq & 32 CampoPrincipe \\
15 & 37 TorringtonSq & 17 Noorderplantsoen & 31 PlazaBibRambla \\
16 & 33 SanMarco & 16 SanMarco & 30 EustonTap \\
17 & 22 CamdenTown & 15 CamdenTown & 27 RegentsParkFields \\
18 & 16 EustonTap & 14 EustonTap & 27 RegentsParkJapan \\

\end{longtable}

\textsubscript{Source:
\href{https://MitchellAcoustics.github.io/J2401_JASA_SSID-Single-Index/notebooks/SingleIndex-Code.ipynb.html\#cell-tbl-ex-spis}{SPI
- Defining bespoke and archetypal context-dependent Soundscape
Perception Indices}}

\textbf{\emph{--- Complete to here ---}}

\section{Expanding the SPI framework}\label{expanding-the-spi-framework}

Section~\ref{sec-method} has defined and demonstrated the foundational
methodology for calculating an SPI score. This included how to: define
and sample a target distribution; prepare the test and target
distributions for comparison using the KS distance metric; and convert
this into an SPI score. Based on this foundation, we develop several
formal aspects of the SPI framework: multi-target SPIs, bespoke and
archetypal targets, and methods for empirically deriving targets.

\begin{longtable}[]{@{}ccc@{}}
\toprule\noalign{}
SPI class & Target class & \\
\midrule\noalign{}
\endhead
\bottomrule\noalign{}
\endlastfoot
& Bespoke & Archetypal \\
Single & & \\
Multi-target & & \\
\end{longtable}

\subsection{Multi-target SPIs (mSPI)}\label{sec-mSPI}

Multi-target SPIs (mSPI) are categories of soundscape types made up of
sets of these targets.

As an example, imagine trying to define a soundscape perception index
that could be applied across an entire city. A single index is
insufficient, because each type of place within the city (e.g.~parks,
plazas, residential areas) has different requirements for its
soundscape. Therefore, each place type would need its own soundscape
target.

An mSPI is made up of sets of targets; in the example given above, these
sets of targets would correspond to different types of places within the
city (e.g.~a single target for parks, a target for plazas etc.). When
applying this ``urban typology'' SPI, the soundscape of each location
being assessed would be scored against its relevant target (i.e how well
does a specific park perform in comparison to an archetypal park
target). This results in a single score for each location that can be
compared against all other locations, regardless of whether or not they
are the same type of place, allowing for different soundscapes to be
compared on a common scale. This system ensures that context (in this
case, the typology of a space) is brought into the assessment, allowing
soundscapes to be scored against the most appropriate target. Enabling
these context dependent assessments to be expressed on a common scale
can facilitate additional use cases such as soundscape mapping, which
requires a single scale to be applied across an entire city.

This \(\text{mSPI}_{type}\) made up of e.g.~parks, plazas etc. is just
one example of an application of multi-target archetypal SPI. Other
examples could include a demographics SPI, where different targets are
set for respondents from different demographic groups, or a ``use case''
SPI with different targets set for different intended purposes of spaces
(e.g.~recreation, restoration, socialising). We encourage users of the
SPI to define both their own single archetype targets that can be added
these suites of targets for use by others, and their own new sets of
archetypes.

To demonstrate the practical implementation of the SPI framework and
provide an example of empirically-defined targets, a case study focused
on defining a typology-based SPI for public spaces is presented. This
case study utilizes data from the International Soundscape Database
(ISD) \citep{Mitchell2021International}, a comprehensive collection of
soundscape recordings and associated listener evaluations gathered under
the SSID Protocol \citep{Mitchell2020Soundscape}. The SSI Protocol was
specifically designed to capture the multi-dimensional nature of
soundscape perception, employed a rigorous methodology for collecting
and analysing data from diverse public spaces according to the
standardized methods in ISO 12913-2 \citep{ISO12913Part2}.

\subsubsection{Space Typologies}\label{space-typologies}

The case study focuses on defining an archetypal SPI for public spaces,
with a particular emphasis on space typologies. The concept of space
typologies is rooted in the idea that different types of public spaces,
such as parks, squares, streets, and plazas, exhibit distinct acoustic
characteristics and elicit unique perceptions from their users. By
defining archetypal SPIs for these space typologies, it becomes possible
to establish a standardized framework for evaluating and comparing
public spaces based on their soundscape quality.

The ISD encompasses a diverse range of public space typologies,
including urban parks, city squares, public walkways, and busy streets.
These typologies serve as the basis for defining archetypal targets and
calculating the corresponding SPIs.

\subsubsection{\texorpdfstring{Defining
\(SPI_{type}\)}{Defining SPI\_\{type\}}}\label{defining-spi_type}

Using the soundscape circumplex model and the perceptual data from the
ISD, the process of defining the \(SPI_{type}\) for each space typology
involves the following steps:

\begin{enumerate}
\def\labelenumi{\arabic{enumi}.}
\tightlist
\item
  Identifying Archetypal Targets: Based on the available data \ldots{}
  target soundscapes are defined for each space typology, representing
  the `ideal' soundscape perception for that particular type of public
  space.
\item
  Calculated \(SPI_{type}\) for each test location: Using the procedure
  given above, the circumplex distribution of each test location is
  compared against the target distribution for its respective space
  typology.
\end{enumerate}

The resulting \(SPI_{type}\) values provide a quantitative measure of
soundscape quality for each space typology, enabling comparisons and
benchmarking across different public spaces. By comparing each test
soundscape against the appropriate target for its typology, the SPI is
able to account for the different contexts and purposes of the
typologies. By using a consistent scoring methodology, SPI then allows
these scores to be combined and considered together, as a single
\(SPI_{type}\) score.

\subsection{Types of Targets}\label{sec-targets}

The SPI framework introduces two distinct types of targets: bespoke
targets and archetypal targets, each serving a unique purpose in the
index development process.

\subsubsection{Bespoke Targets}\label{bespoke-targets}

Bespoke targets are essentially a direct application of the foundational
method described above. Bespoke targets are tailor-made for specific
projects, reflecting the desired soundscape perception for a particular
application. These targets can be defined by stakeholders, designers,
policymakers, or decision-makers based on their unique requirements,
objectives, and constraints. This flexibility allows the SPI for a
specific project to be tailored to the desire of the stakeholders for
how that specific soundscape should function. It can also provide a
consistent and quantifiable baseline for scenarios like a soundscape
design contest wherein a target is specified and provided to all
participants in the contest and the winning proposal is the design with
the highest SPI score when assessed against that target. Stakeholders
could use various methods to decide on a target, subject to the
requirements of their project or use case. For example, it could be
co-created with other stakeholders or space users, based on trying to
match the soundscape of a previous project, or entirely arbitrary.

\subsubsection{Archetypal Targets}\label{archetypal-targets}

In contrast to bespoke targets, archetypal targets represent
generalized, widely recognized soundscape archetypes which transcend
specific applications or projects. These archetypes serve as reference
points and enable comparisons across different domains and use cases.
Essentially an archetypal target is a target that has been empirically
defined to encapsulate the ideal of a particular type of soundscape.

\subsection{Empirically defining a target based on soundscape
ranking}\label{empirically-defining-a-target-based-on-soundscape-ranking}

Absent from the above methodology has been an exploration of how to
actually arrive at a target based on empirical evidence. While bespoke
targets make the SPI framework incredibly flexible, able to score
against an effectively infinite set of design goals, archetypal targets
intended to be used as a reference standard should have some empirical
foundation. One method for doing this is to arrive at a ranking of
soundscape quality through some other method (which would typically be
much more involved than a simple SCM survey) then derive a target which,
when scored against the soundscapes, produces the same rank order.

Effectively this is an optimization task. We consider the Spearman rank
correlation coefficient between the provided ranking and the SPI ranking
to be an error term, then learn the MSN parameters to which optimize
this error term.

Need to consider both the Spearman rank coefficient and the SPI score
itself. Through our testing, only optimizing on the rank correlation
regularly produced targets which, while they did result in the desired
ranking, were in no way representative of the soundscapes in question.
We therefore aim to optimize for both a consistent soundscape ranking
and for a high SPI score for the top-ranked soundscapes.

We apply an evolutionary multiobjective optimization named NSGA-II
\citep{Deb2014Evolutionary}.

Defining the optimization problem:

\begin{itemize}
\tightlist
\item
  max \(r(ranks_{quality}, ranks_{target})\)
\item
  max \(mean(SPI_{target}(X_i))\)
\end{itemize}

where \(r\) is the rank correlation coefficient, \(ranks_{quality}\) and
\(ranks_{target}\) are the ranks of the quality and target values, and
\(SPI_{target}(X_i)\) is the SPI for a given target on the data for the
\(i\)-th location. Therefore we are trying to achieve the best
correlation between the desired ranking and the ranking produced by
\(SPI_{target}\) \emph{and} to achieve the highest mean
\(SPI_{target}\).

\(ranks_{quality}\) is pre-defined. \(ranks_{target}\) is calculated by
sorting the target values and assigning ranks to them. \(SPI_{target}\)
is calculated for each location and target.

\section{Discussion}\label{discussion}

The development of bespoke and archetypal context-dependent Soundscape
Perception Indices (SPIs) represents a significant step towards enabling
more comprehensive and effective applications of the soundscape
approach. By providing a unified framework for defining these indices,
the potential for quantifying and comparing soundscape quality across
diverse contexts and applications is unlocked, while still ensuring that
the multi-dimensional and context-driven aspects of soundscape quality
are considered.

The proposed framework offers several key advantages. First, it
acknowledges the inherent context-dependent nature of soundscape
perception, allowing for the creation of indices tailored to specific
use cases or design goals through the use of bespoke targets. This
flexibility ensures that the resulting SPIs accurately capture the
desired soundscape perception for the given application, enabling
targeted interventions and optimisations.

Second, the inclusion of archetypal targets facilitates
cross-comparisons and benchmarking, enabling a common language and
understanding of soundscape quality across different domains. By
calculating the distance between a given soundscape and these widely
recognized archetypes, stakeholders can identify areas for improvement
and prioritize interventions accordingly, aligning their efforts with
collectively recognized standards of desirable or undesirable
soundscapes.

The case study presented in this article, focusing on the development of
a typology-based SPI for public spaces, demonstrates the practical
applicability of the framework. By leveraging data from the
International Soundscape Database (ISD) and the SSID Protocol,
archetypal targets for various space typologies were defined, and the
corresponding \(SPI_{type}\) values were calculated. These indices
provide a quantitative measure of soundscape quality for each typology,
enabling comparisons and informing decision-making processes related to
the management and improvement of public spaces.

As stated in \#sec-intro \ldots{}

\citep[Fig.6]{Kogan2018Green}, in fact displays a startlingly similar
concept, showing the locations of the three categories of traffic noise
dominance (`traffic noise', `balanced', and `natural') plotted in the
circumplex perceptual model. It can be clearly seen in this plot that
the GSI categories create their own clusters within the circumplex.

Although it is expected that the target distribution would usually
represent the ideal or goal soundscape perception, it is also possible
to define target distributions that represent undesirable or suboptimal
soundscape perceptions. For instance, in a soundscape mapping context,
it may be beneficial to map and identify chaotic soundscapes across a
city in order to better target areas for soundscape interventions. In
this case, the target distribution would be set in the chaotic quadrant
and a higher SPI would indicate a closer alignment with the target
distribution. This flexibility allows the SPI to be applied to a wide
range of contexts and applications, enabling the quantification and
comparison of soundscape quality across diverse scenarios.

\subsubsection{Data Source}\label{data-source}

The SPI framework is designed to accommodate a wide range of data
sources, including both objective measurements and subjective
evaluations. This flexibility enables the framework to be applied to
diverse contexts and applications, ranging from urban soundscapes to
natural environments, public spaces, and indoor settings.

\subsection{Applying a Bespoke SPI}\label{applying-a-bespoke-spi}

\section{Conclusion}\label{conclusion}

The introduction of bespoke and archetypal context-dependent Soundscape
Perception Indices (SPIs) represents a significant advancement in the
field of soundscape research and application. By providing a unified
framework for defining these indices, a more comprehensive and efficient
approach to quantifying and comparing soundscape quality across diverse
contexts is enabled.

The proposed framework addresses the existing gap in quantifying
multi-dimensional soundscape perception, facilitating a broader
application of the soundscape approach in areas such as urban planning,
environmental management, acoustic design, and policy development.
Through the creation of bespoke indices tailored to specific design
goals and the utilization of archetypal targets for benchmarking, this
framework empowers stakeholders and decision-makers to make informed
choices and prioritize soundscape improvements aligned with their unique
objectives and constraints.

Furthermore, the grounding of the SPI framework in the soundscape
circumplex model ensures a robust theoretical foundation, capturing the
multi-dimensional nature of soundscape perception. The use of a distance
metric enables quantitative assessments and comparisons, fostering a
common language and understanding of soundscape quality across different
domains. This shared understanding facilitates knowledge exchange,
collaborative efforts, and the development of best practices within the
field.

The case study presented in this article, focused on defining a
typology-based SPI for public spaces, demonstrates the practical
applicability of the framework and highlights its potential for enabling
more effective and context-sensitive soundscape management strategies.
By leveraging data from the International Soundscape Database (ISD) and
the SSID Protocol, archetypal targets for various public space
typologies were defined, and the corresponding \(SPI_{type}\) values
were calculated, providing a quantitative measure of soundscape quality
that can inform decision-making processes and guide interventions.

As the SPI framework continues to be explored and refined, future
research should focus on validating and expanding the range of
archetypal targets, as well as investigating the potential for
incorporating additional dimensions and factors that influence
soundscape perception. The integration of emerging technologies, such as
virtual and augmented reality, may also provide new avenues for
immersive soundscape evaluation and index development.

Additionally, the application of the framework in diverse real-world
scenarios, ranging from urban planning and environmental management to
acoustic design and policy development, will provide valuable insights
and contribute to the ongoing refinement and adaptation of the SPI
framework. Collaboration with stakeholders, end-users, and experts from
various domains will be crucial in ensuring the framework's relevance
and applicability across a wide range of contexts.

Furthermore, the development of standardized data collection protocols
and the establishment of comprehensive soundscape databases will be
essential for the widespread adoption and effective implementation of
the SPI framework. Initiatives focused on promoting data sharing,
interoperability, and open access to soundscape data can significantly
facilitate the creation and validation of new indices, fostering a more
collaborative and data-driven approach to soundscape research and
management.

Ultimately, the introduction of bespoke and archetypal context-dependent
Soundscape Perception Indices represents a significant stride towards a
more holistic and nuanced understanding of our acoustic environments,
paving the way for more informed decision-making and enhancing the
overall quality of life in our built and natural environments. By
empowering stakeholders with the ability to quantify and compare
soundscape quality, new avenues are unlocked for targeted interventions,
strategic planning, and the creation of soundscapes that are not only
acoustically optimal but also deeply resonant with the diverse needs and
perceptions of individuals and communities.


\renewcommand\refname{References}
  \bibliography{FellowshipRefs-biblatex.bib}


\end{document}
